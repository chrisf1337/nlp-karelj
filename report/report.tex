%
% File acl2016.tex
%
%% Based on the style files for ACL-2015, with some improvements
%%  taken from the NAACL-2016 style
%% Based on the style files for ACL-2014, which were, in turn,
%% Based on the style files for ACL-2013, which were, in turn,
%% Based on the style files for ACL-2012, which were, in turn,
%% based on the style files for ACL-2011, which were, in turn,
%% based on the style files for ACL-2010, which were, in turn,
%% based on the style files for ACL-IJCNLP-2009, which were, in turn,
%% based on the style files for EACL-2009 and IJCNLP-2008...

%% Based on the style files for EACL 2006 by
%%e.agirre@ehu.es or Sergi.Balari@uab.es
%% and that of ACL 08 by Joakim Nivre and Noah Smith

\documentclass[11pt]{article}
\usepackage{acl2016}
\usepackage{times}
\usepackage{url}
\usepackage{latexsym}

\aclfinalcopy % Uncomment this line for the final submission
%\def\aclpaperid{***} %  Enter the acl Paper ID here

%\setlength\titlebox{5cm}
% You can expand the titlebox if you need extra space
% to show all the authors. Please do not make the titlebox
% smaller than 5cm (the original size); we will check this
% in the camera-ready version and ask you to change it back.

\newcommand\BibTeX{B{\sc ib}\TeX}

\title{A Dependency-Parser-Based Approach to Converting Natural Language English Sentences to Karel J.\ Robot Code}

\author{
    % First Author \\
    % Affiliation / Address line 1 \\
    % Affiliation / Address line 2 \\
    % Affiliation / Address line 3 \\
    % {\tt email@domain} \\\And
    % Second Author \\
    % Affiliation / Address line 1 \\
    % Affiliation / Address line 2 \\
    % Affiliation / Address line 3 \\
    % {\tt email@domain} \\}
    Christopher Fu \\
    \texttt{christopher.fu@yale.edu} \\}
\date{}

\begin{document}
\maketitle
\begin{abstract}
  This document contains the instructions for preparing a camera-ready
  manuscript for the proceedings of ACL-2016. The document itself
  conforms to its own specifications, and is therefore an example of
  what your manuscript should look like. These instructions should be
  used for both papers submitted for review and for final versions of
  accepted papers.  Authors are asked to conform to all the directions
  reported in this document.
\end{abstract}

\section{Introduction}
Programming students often find that when they are first learning how to to program, accurately
translating what they want their program to do into actual code is harder than expected. Newell and
Card~\shortcite{Newell:1985aa} suggest that one factor contributing to this difficulty is that most
popular programming languages are not particularly designed with human-computer interaction issues
in mind. Some programming languages more easily lend themselves to a natural translation between
programmer intent and actual code, a concept that Green and Petre~\shortcite{Green:1996aa} term
\emph{closeness of mapping}.



\section{Approach}


% include your own bib file like this:
%\bibliographystyle{acl}
%\bibliography{acl2016}
\bibliography{report}
\bibliographystyle{acl2016}

\appendix

\section{Supplemental Material}
\label{sec:supplemental}
ACL 2016 also encourages the submission of supplementary material
to report preprocessing decisions, model parameters, and other details
necessary for the replication of the experiments reported in the
paper. Seemingly small preprocessing decisions can sometimes make
a large difference in performance, so it is crucial to record such
decisions to precisely characterize state-of-the-art methods.

Nonetheless, supplementary material should be supplementary (rather
than central) to the paper. It may include explanations or details
of proofs or deriations that do not fit into the paper, lists of
features or feature tempates, sample inputs and outputs for a system,
pseudo-code or source code, and data. (Source code and data should
be separate uploads, rather than part of the paper).

The paper should not rely on the supplementary material: while the paper
may refer to and cite the supplementary material will be available to the
reviewers, they will not be asked to review the
supplementary material.

Appendices (i.e. supplementary material in the form of proofs, tables,
or pseudo-code) should come after the references, as shown here. Use
\verb|\appendix| before any appendix section to switch the section
numbering over to letters.

\section{Multiple Appendices}
\dots can be gotten by using more than one section. We hope you won't
need that.

\end{document}
